\documentclass[letterpaper]{article}
\usepackage{natbib,alifexi}

\title{Project Computational Game Theory\\Evolution of All-or-None Strategies in Repeated Public Goods Dilemmas}
\author{Ruben Vereecken, Yoni Pervolarakis \and Laurens Hernalsteen \\
\mbox{}\\Vrije Universiteit Brussel, \\Pleinlaan 2, B-1050 Brussels, BELGIUM\\
{\texttt{\{rvereeck,ypervola,lhernals\}@vub.ac.be}}}


\begin{document}
\maketitle

\begin{abstract}
Engaging in repeated group interactions such as \textit{Public Good Games}  (\textbf{PGG}), groups of individuals may contribute to a common pool and subsequently share their resources. In this paper we will recreate the research and results from the paper  \textit{Evolution of All-or-None Strategies in Repeated Public Goods Dilemmas}  \citep{project}. Studying evolutionary dynamics, where individuals behave on what they observed in the previous round, a simple strategy \textit{All-Or-None} \textbf{(AON)} will emerge. AON consists of cooperating only after an unanimous group behavior. This strategy will prove its robustness by the presence of errors and different group sizes.
\textit{\textbf{Short summary of results....}}


\end{abstract}

\section{Introduction}
%citet (for more) or citep *for1)?



\section{Methods}
%\begin{figure}[t]
%\begin{center}
%\includegraphics[width=2.1in,angle=-90]{fig1.eps}
%\caption{``Energies'' (inferiorities) of strings in a first-order
%  phase transition with latent heat $\Delta\epsilon$.}
%\label{fig1}
%\end{center}
%\end{figure}
% In Fig.~\ref{fig2}




\section{Results}






\section{Discussion}
Something something

\section{Acknowledgments}

Help from Flávio L. Pinheiro....


\footnotesize
\bibliographystyle{apalike}
\bibliography{biblio}


\end{document}
